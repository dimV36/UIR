\documentclass[tikz,convert]{standalone}
\usepackage[utf8]{inputenc}

\usepackage[T2A]{fontenc}
\usepackage[russian,english]{babel}
\usepackage{xcolor,colortbl}
\usepackage{array}
\usepackage{tikz}

\usetikzlibrary{shapes, arrows}
\usetikzlibrary{positioning}

\renewcommand{\arraystretch}{1.2}
\newcolumntype{C}[1]{>{\centering\arraybackslash}p{#1}}

\begin{document}

\begin{tikzpicture} [
    auto,
    block/.style    = { rectangle, thick, draw = blue, 
                        text width=22.3em, text centered,
                        rounded corners, minimum height=2em,
                        scale = 0.75, 
                        %font=\ttfamily, 
                        },
                        ->,>=stealth',shorten >=1pt,auto,
                        thick,
  ]
  % Define nodes in a matrix

\node[block] (ca) {
	\begin{tabular}{C{7.4cm}}
		\rowcolor{blue!30}
		\textbf{\begin{tabular}[c]{@{}c@{}}\Large Удостоверяющий центр\\ \large IP: 192.168.100.2\end{tabular}} \\                                                   
	\end{tabular}
};

\node[block, below of = ca, node distance = 15cm] (client) {
	\begin{tabular}{ll}
		\rowcolor{blue!30}
		\multicolumn{2}{c}{\textbf{\begin{tabular}[c]{@{}c@{}} \LargeКлиент\\ \large IP: 192.168.100.3\end{tabular}}} \\
		\rowcolor{blue!15}
		\multicolumn{2}{c}{\largeПользователи ОС:}                                                            \\
		\rowcolor{blue!10}
		\texttt{user1}                           & \texttt{user\_u:user\_r:user\_t:s0}                                    \\
		\rowcolor{blue!10}
		\texttt{user2}                           & \texttt{user\_u:user\_r:user\_t:s0-s2}                                 \\
		\rowcolor{blue!10}
		\texttt{user3}                           & \texttt{user\_u:user\_r:user\_r:s1-s3:c0.c10}                         
	\end{tabular}
};

\node[block] (postgresql) [below left of=ca, node distance = 15cm] {
	\begin{tabular}{C{7.4cm}}
		\rowcolor{blue!30}
		\textbf{\begin{tabular}[c]{@{}c@{}}\Large PostgreSQL\\ \Large IP: 192.168.100.4\end{tabular}} \\
		\rowcolor{blue!15}
		\largeПользователи СУБД:                                                          \\
		\rowcolor{blue!10}
		\texttt{user1}                                                                       \\
		\rowcolor{blue!10}
		\texttt{user2}                                                                       \\
		\rowcolor{blue!10}
		\texttt{user3}                                                                      
	\end{tabular}
};

  \path[every node/.style={font=\sffamily\small}]
    (client) edge [bend right] node [left, align = center] {Запрос к БД} (postgresql)
    (client) edge [bend right] node [right, align = center] {CSR \\клиента} (ca)
    (postgresql) edge [bend right] node [left, align = center] {Ответ БД} (client)
    (postgresql) edge [bend left] node [right, align = center] {CSR \\сервера} (ca.west)
    (ca) edge [bend right] node [right, align = center] {Сертификат \\клиента} (client)
    (ca.west) edge [bend left] node [left, align = center] {Сертификат \\сервера СУБД} (postgresql);
  
\end{tikzpicture}

\end{document}
