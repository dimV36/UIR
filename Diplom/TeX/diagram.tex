\documentclass{article}
\usepackage[utf8]{inputenc}

\usepackage[T2A]{fontenc}
\usepackage[russian,english]{babel}

\usepackage{tikz}
\usetikzlibrary{shapes, arrows}
\usetikzlibrary{positioning}

\begin{document}

\begin{tikzpicture} [
    auto,
    decision/.style = { diamond, draw=blue, thick, fill=blue!20,
                        text width=5em, text badly centered,
                        inner sep=1pt, rounded corners,
                        scale = 0.75 },
    block/.style    = { rectangle, draw=blue, thick, 
                        fill=blue!20, text width=10em, text centered,
                        rounded corners, minimum height=2em,
                        scale = 0.75 },
    line/.style     = { draw, thick, ->, shorten >=2pt, scale = 0.75 },
    cloud/.style    = { ellipse, draw, fill=red!20, text width=5em, text badly centered, inner sep=1pt, scale = 0.75 },
  ]
  % Define nodes in a matrix
  \matrix [column sep=3mm, row sep=5mm] {
                    & \node [cloud] (start) {Открытие сессии клиента};                                                 & \\
                    & \node [decision] (dsacondition) {ЭЦП создана?};                                                  & \\
                    & \node (null1) {}; & \node [block] (dsacreate) {Создание ЭЦП};                                    & \\
                    & \node (null2) {};                                                                                & \\
                    & \node [block] (phase1) {Создание закрытого ключа клиента};                                       & \\
                    & \node [block] (phase2) {Создание CSR клиента};                                                   & \\
                    & \node [block] (phase3) {Подпись запроса ЭЦП};                                                    & \\
                    & \node [block] (phase4) {Отправка подписанного CSR клиента на УЦ};                                & \\
                    & \node [decision] (dsaverify) {ЭЦП валидна?};                                                     & \\
                    & \node [block] (phase5) {Выпуск сертификата клиента};                                             & \\
                    & \node [block] (phase6) {Отправка сертификата клиенту};                                           & \\
                    & \node [decision] (newlevelcondition) {Требуется перейти на новый уровень?};                      & \\
                    & \node [cloud] (finish) {Окончание работы алгоритма}; & \node [cloud] (error) {Выход с ошибкой};  & \\
  };
  \begin{scope} [every path/.style=line]
    \path (start)        -- (dsacondition);
    \path (dsacondition) -- node [near start] {Да} (phase1);
    \path (dsacondition) -| node [near start] {Нет} (dsacreate);
    \path (dsacreate) |- (null2);
    \path (phase1) -- (phase2);
    \path (phase2) -- (phase3);
    \path (phase3) -- (phase4);
    \path (phase4) -- (dsaverify);
    \path (dsaverify) -- node [near start] {Да} (phase5);
    \path (dsaverify) -| node [near start] {Нет} (error);
    \path (phase5) -- (phase6);
    \path (phase6) -- (newlevelcondition);
    \path (newlevelcondition) -- node [near start] {Нет} (finish);
    \path (newlevelcondition) --++ (-3,0) node [near start] {Да} |- (null1);
  \end{scope}
  
\end{tikzpicture}

\end{document}
